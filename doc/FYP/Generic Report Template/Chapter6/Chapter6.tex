% !TEX root =  ../Dissertation.tex
\chapter{Result}

A total of 15 participants took part in the experiment, with their consent. Participants came from both China (CN), Hong Kong (HK), and the United Kingdom (GB), with some identifying differently for their cultural background.
\section{Participant Demographics}
\begin{itemize}
\item 5 were born in China.
\item 4 were born in Hong Kong.
\item 6 were born in the United Kingdom.
\end{itemize}
\section{Long-time residence were}
\begin{itemize}
\item 3 resided long-term in Hong Kong.
\item 10 resided long-term in the United Kingdom.
\item 2 resided long-term in China.
\end{itemize}

While all of them currently reside in the United Kingdom (GB), it's notable that one of the participants, despite being born in and having spent a significant amount of time in the United Kingdom, identifies their cultural background as Malaysian (MY). This observation underscores the distinction between geographical origin and cultural identification.

\begin{table}[htbp]
\centering
\label{tab:pre_experiment_survey}
\begin{tabular}{|l|c|c|}
\hline
\textbf{Category} & \textbf{Frequency} & \textbf{Percentage (\%)} \\ \hline
\textbf{Consent Rate} & \multicolumn{2}{c|}{100\%} \\ \hline
\multirow{3}{*}{\textbf{Country of Birth}} & CN (China) & 33.3 \\
 & HK (Hong Kong) & 33.3 \\
 & GB (United Kingdom) & 33.3 \\ \hline
\multirow{3}{*}{\textbf{Long-Time Country}} & CN (China) & 26.7 \\
 & HK (Hong Kong) & 26.7 \\
 & GB (United Kingdom) & 46.7 \\ \hline
\textbf{Current Country of Residence} & GB (United Kingdom) & 86.7 \\
 & HK (Hong Kong) & 13.3 \\ \hline
\multirow{4}{*}{\textbf{Cultural Background Identification}} & CN (Chinese) & 46.7 \\
 & HK (Hong Kong) & 40.0 \\
 & GB (British) & 33.3 \\
 & MY (Malaysian) & 6.7 \\ \hline
\end{tabular}
\caption{Summary of Pre-Experiment Survey Data}
\end{table}

\section{Robot Preference}

Participants were asked about their robot preference:
\begin{itemize}
\item 5 participants preferred the first robot due to its behaviours adapting according to their backgrounds.
\item 10 participants preferred the second default English robot because they found its language more familiar and understandable.
\end{itemize}
These preferences are associated with their backgrounds and familiarity with the English language.

\begin{table}[htbp]
\centering
\label{tab:robot_preferences}
\begin{tabular}{|l|c|p{6cm}|}
\hline
\textbf{Robot} & \textbf{Participants} & \textbf{Reason for Preference} \\ \hline
First Robot & 5 & Behaviors adapt according to participants' backgrounds \\ \hline
Second Default English Robot & 10 & Language is more familiar and understandable \\ \hline
\end{tabular}
\caption{Robot Preferences}
\end{table}

\section{Robot Behaviours}

Participants indicated their liked behaviors in the robots:
\begin{itemize}
\item All participants appreciated the "Conversation Language" of the robots.
\item 5 participants specifically highlighted the "Proximity" as a liked behavior in the first robot.
\end{itemize}

\begin{table}[h]
\centering
\begin{tabular}{|l|l|}
\hline
\textbf{Robot Behaviours} & \textbf{Number of Participants} \\
\hline
Conversation Language & All \\
\hline
Proximity (First Robot) & 5 \\
\hline
\end{tabular}
\caption{Participants' liked behaviors in the robots}
\label{tab:robot_behaviours}
\end{table}

\section{Robot Performance Ratings}

Participants' performance ratings varied, demonstrating different perspectives on how they evaluated the robots:
\begin{itemize}
\item 9 participants rated the robots as "Good".
\item 2 participants rated the robot's performance as "Average".
\item 1 participant (Participant 15) rated the robot's performance as "Poor" due to language barriers as they could not understand Chinese.
\end{itemize}

\begin{table}[h]
\centering
\begin{tabular}{|l|l|}
\hline
\textbf{Performance Rating} & \textbf{Number of Participants} \\
\hline
Good & 9 \\
\hline
Average & 2 \\
\hline
Poor (due to language barriers) & 1 (Participant 15) \\
\hline
\end{tabular}
\vspace{1em}
\caption{Participants' performance ratings of the robots}
\label{tab:performance_ratings}
\end{table}

