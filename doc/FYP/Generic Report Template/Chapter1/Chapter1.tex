% !TEX root =  ../Dissertation.tex

\chapter{Introduction}

The rapid integration of robotic agents into human-centric environments has sparked a burgeoning discourse on Human-Robot Interaction (HRI). No longer confined to industrial settings as task-oriented automatons, modern robots are increasingly merging into the fabric of diverse sociocultural landscapes. A prime example of this integration is the multifaceted roles robots are assuming, from guides in the Lincoln Centre for Autonomous Systems \cite{lincoln2018}, to instrumental aids in education and social care \cite{soriano2022}. Beyond operational functionalities, the emerging domain of social robotics calls for nuanced human-robot interactions that respect the complex mosaic of human culture.

Recognising the importance of cultural dimensions, this project aims to explore the impact of cultural differences on HRI, with a specific focus on the greeting behaviours of culturally adaptive robots within an OfficeBots environment, simulated through the ROS4HRI framework. As contemporary workspaces evolve into multicultural environments, the need to calibrate robotic behaviours---starting with the fundamental act of greeting---to align with diverse cultural etiquettes and practices has become increasingly critical. Consequently, this study seeks to distil the intricacies of culturally adaptive greetings by robots, aiming to enhance both the effectiveness and sociocultural acceptance of robotic agents in multi-ethnic office settings.

The ROS4HRI framework, currently under peer review, serves as the foundation for this investigation. Designed to introduce uniform interfaces and conventions for HRI within the Robot Operating System (ROS), ROS4HRI aims to address the limitations of previous implementations, promoting code reusability, experiment replicability, and knowledge sharing across the HRI research community. The proposed research leverages the potential of ROS4HRI to simulate spatial interactions in OfficeBots---an interactive 3D simulation designed for HRI research and education. Within this digital microcosm, researchers can instantiate, customise, and control robotic avatars, setting the stage for empirical user studies focused on robots' culturally attuned communication exchanges with participants.

While the unification of the HRI research framework is promising, it is disappointing that it lacks consideration of the cultural aspects of robots. In response to this, the project aims to extend the ROS4HRI framework to incorporate and consider culturally aware robots by developing a CulturalHRI framework that invites researchers to collaboratively contribute.