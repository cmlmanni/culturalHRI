% !TEX root =  ../Dissertation.tex

\chapter{Conclusion}

In conclusion, this project has developed a culturally aware robotic framework that demonstrates significant potential for enhancing human-robot interaction (HRI) research using nationality identity/national culture as case study. By leveraging the foundational principles of the Robot Operating System (ROS) and extending its capabilities with ROS4HRI, the project created a flexible platform enabling further research in accommodating cultural nuances in robotic behaviour.

The framework's modular design and adaptable nature, made possible by the use of factory pattern, enable it to serve as a valuable testbed for future studies exploring the complex interplay between culture and HRI dynamics. While the framework exhibits promising utility, it is essential to acknowledge its limitations and the ongoing challenges associated with accurately representing and adapting to diverse cultural contexts.

Through ongoing research and interdisciplinary collaboration, this project aims to address these challenges and further refine the framework's capabilities. This will pave the way for culturally aware robots to integrate into various social settings, enriching human experiences and fostering meaningful interactions across cultural divides.


