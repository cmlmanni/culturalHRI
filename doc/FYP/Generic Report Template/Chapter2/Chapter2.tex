% !TEX root =  ../Dissertation.tex

\chapter{Literature Review}

This literature review critically evaluates pertinent studies in the domain of Human-Robot Interaction (HRI), with an emphasis on the cultural adaptability of robots. The increasing integration of robots in diverse sectors signals the importance of investigating their cultural competence. The review encompasses various dimensions of HRI, including investigations of social robots, implications of bias in artificial intelligence, the field of cultural robotics, and the requirement for an open-source framework tailored to Cultural HRI research. Pulling from these themes, the review frames the present landscape of HRI research and acknowledges potential future areas of exploration. It should be noted that the development of culturally adaptable robots extends beyond making headway in technical prowess - it necessitates an in-depth examination into issues such as bias, ethics, and societal implications inherent to AI technologies.

\section{Cultural Adaptability in Human-Robot Interaction (HRI)}
The integration of robots into various aspects of daily life is a growing trend, transcending their conventional roles as mere machines into more socially interactive entities \cite{soriano2022}. The effectiveness of robots in these roles hinges on their ability to adapt to the cultural backgrounds of the individuals they interact with \cite{lim2020}. As such, cultural adaptability has become a paramount concern in Human-Robot Interaction (HRI).

\section{HRI Research and Social Robots}
Tobis et al.'s study \cite{tobis2022} investigates the acceptance of service robots in eldercare, emphasizing the role of social support, relational coordination, and technology readiness. Cultural adaptability becomes paramount in contexts such as eldercare, where robots must cater to the specific cultural needs and expectations of elderly users. Understanding how cultural factors influence robot acceptance is crucial for the design and deployment of robots in diverse settings. Bartneck et al.'s work \cite{bartneck2008}, on the other hand, introduces measurement instruments for anthropomorphism, animacy, likeability, perceived intelligence, and perceived safety of robots. These instruments provide a structured approach to assess user perceptions of robots in culturally diverse contexts. Understanding how users anthropomorphize robots differently across cultures can guide the design of culturally adaptable robots. Mutlu et al. \cite{mutlu2006} explore the significance of human-like gaze behaviour in robots. Gaze behaviour is culturally nuanced and plays a pivotal role in communication. The ability of a robot to adapt its gaze behaviour to the cultural expectations of users can profoundly affect the quality of interactions. While this research aims not to deploy robot during human-robot interaction, it is believed that such metrics are significant to the future development of the field.

\section{The Influence of Bias in Artificial Intelligence on Cultural HRI}
As social robots become increasingly prevalent in real-life settings, claims of their cultural awareness abound. However, given that these robots are powered by AI, it's essential to acknowledge the persistence of cultural bias in AI research. While Lim \cite{lim2020} examined 50 HRI cultural robot research studies, a unifying framework was notably absent, and the research focus did not prioritize responsible robotics.

Scheuerman et al.'s \cite{scheuerman2021} investigation into automated facial analysis serves as a fundamental entry point into the discussion on bias. Their study underscores the role of these technologies in perpetuating historical gender classifications and racial biases, particularly through the essentialisation of facial features. The technology's perpetuation of historical gender classifications by essentialising facial features raises critical questions about the implications of such practices. The authors’ call for critical examination of cultural and historical contexts resonates with the broader theme of understanding the roots of bias in AI.

Similarly, Prates et al. \cite{prates2019} examined machine translation, offering insights into the societal implications of biased algorithmic outputs. Unveiling a notable inclination towards male defaults, this study highlights the societal implications of biased algorithmic outputs, emphasizing the critical need for debiasing techniques in translation tools. As language plays a crucial role in shaping gender bias, the study brings a full circle to the initial exploration of biases in facial analysis.

Gupta, Parra, and Dennehy's \cite{gupta2021} article, on the other hand, provided interesting insight into how individuals' responses to biased AI recommendations unveil how culture – namely collectivism, masculinity, and uncertainty avoidance – shape individuals' responses to biased AI recommendations. This echoes Hagendorff \cite{hagendorff2020}, who emphasized that gender bias not only lies in the production of AI but also in its users, inviting researchers to recognize the intricate interplay between technology and culture.

\section{Cultural Robotics}
The above studies echo with Brandão, Mansouri, and Magnusson's \cite{brandao2022} editorial, written from the perspective of robotics, shed light on the importance of considering the ethical and social implications of AI in general. The editorial attempts to provide a socio-technical perspective on AI's societal impacts with a comprehensive discussion that encompasses physical safety concerns, biased conceptions of gender and race, and the development of responsible and trustworthy robotics and AI. The emphasis on the socio-technical perspective and issues of bias, transparency, and fairness provides a comprehensive lens to evaluate the responsible development and deployment of AI technologies. As the editorial emphasizes the need for a socio-technical perspective, it invites researchers to consider the broader societal impacts and ethical considerations that underpin responsible AI development, going beyond a mere technical perspective.

Lim et al. \cite{lim2020} emphasize the importance of the role of culture in shaping HRI, advocating for a global perspective in robot design and interaction. Through surveying existing literature in the field, Lim et al. concluded that cultural nuances significantly influence how robots are perceived and accepted by users from diverse backgrounds. This is particularly crucial in multicultural environments, where robots are expected to interact with users representing a spectrum of cultural norms. 

As such, Mansouri \cite{mansouri2022} calls for an epistemic analysis of cultural theories in robotics and AI methods, providing a theoretical foundation for understanding the role of culture in human-robot interactions. The exploration of how robots can adapt and learn from cultural interactions over time is a promising avenue for future research. 

\section{Developing an Open-Source Framework for Cultural HRI Research}
Progress has been made in constructing a cohesive framework for HRI research. The ROS4HRI framework, launched in October 2022, establishes standardized topics and conventions explicitly designed for HRI. These focus on "traditional" HRI dimensions such as face detection and body tracking. However, cultural aspects have not received the same level of attention.

In light of the intersections between cultural adaptability in HRI and the implications of bias in AI, there is a pressing need to develop an open-source framework for Cultural HRI research. This framework should integrate insights from both fields and provide guidelines for the responsible development and deployment of culturally adaptable robots.