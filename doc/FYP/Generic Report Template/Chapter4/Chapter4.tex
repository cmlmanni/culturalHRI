% !TEX root =  ../Dissertation.tex

\chapter{Contribution}


Development of a General Framework for Cultural HRI Studies

My research endeavors to pioneer a general framework that empowers researchers to explore the intersection of robotics and culture. This framework is envisioned as a malleable foundation on which various parameters relevant to cultural considerations can be adjusted to accommodate a spectrum of studies. Essential to this development is the integration of simulated environments alongside empirical research to glean insights into the cultural dynamics of HRI.
Simulation Program Customization

Integral to my contribution is the establishment of a robust simulation program that encapsulates the versatility required for user-directed research. This program allows for parameter modification, enabling researchers to tailor their studies to address specific cultural nuances. The potential for customization extends the reach of HRI research, paving the way for a multitude of applications across cultural contexts.
Extension of ROS4HRI to Encompass Cultural Aspects

One of the defining aspects of this contribution is the refinement of ROS4HRI with a focus on cultural attributes. By integrating considerations for nationality and national culture as a case study, my work seeks to infuse these global cultural dimensions into the technical realm of HRI. This extended functionality converges the technical capability of ROS4HRI with the sensitive, often subjective, elements that define cultural interaction and perception.
By unifying these existing tools under the purview of culture-focused research, my work elevates the study of HRI onto a universally applicative platform, fostering exploration, and understanding of how robots can be made more culturally congruent and, thus, more adept at serving alongside humans in a multiplicity of environments.
