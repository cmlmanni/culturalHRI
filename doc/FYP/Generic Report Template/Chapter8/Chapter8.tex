% !TEX root =  ../Dissertation.tex

\chapter{Evaluation}

Expanding on the Discussion, this section assesses the framework's usefulness for ongoing cultural HRI research and identifies potential limitations and extensions.

\section{Critical Evaluation of the Framework}

Examining the framework in action reveals areas for refinement, particularly in how cultural elements are represented by the robot. Clearer delineation of culture-specific behaviours could improve recognizability and enhance the educational value of interactions. Additional visual or auditory cultural cues may be necessary to make implicit cultural references explicit to all users.

\section{Limitations}

Despite the potential of the framework, it's important to recognise its limitations. Cultural expressions are influenced by probabilistic tendencies rather than deterministic rules, meaning that certain assumptions may not align with every individual within a cultural group.

While the framework offers significant advancements in culturally aware robotics, several challenges warrant consideration:

\begin{itemize}
\item Risk of Cultural Stereotyping
\item Accuracy and Recognition
\item Continuous Update
\item Complexity
\item Resistance to Robot Personalisation
\end{itemize}

In the following, each of these limitations will be discussed in detail:

\subsection{Risk of Cultural Stereotyping}

Although the overarching aim is to foster respectful and personalised interactions, the risk of inadvertent cultural stereotyping cannot be overlooked. For instance, not every individual will align with the gender communication styles identified, and not everyone from a specific racial or ethnic group would adhere straightforwardly to the associated cultural norms. Employing broad cultural tendencies as rules, rather than guidelines, could lead to overgeneralizations and potentially offend users.

\subsection{Accuracy and Recognition}

The challenge of accurately recognizing and interpreting a user's gender and cultural cues solely based on interaction is a monumental task. Any misinterpretations can lead to inappropriate behaviour by the robot, resulting in user discomfort.

\subsection{Continuous Update}

Cultures are not static entities but evolve and fluctuate over time, influenced by various socioeconomic factors. Ensuring that the framework remains updated with these cultural transitions presents a considerable challenge.

\subsection{Complexity}

The inherent complexity of human culture, with its myriad identities, nuances, and norms, may pose significant obstacles to replicating fully in an artificial intelligence system. While it may be feasible to achieve a general understanding and adaptability, capturing the intricate details of every sub-culture may remain elusive.

\subsection{Resistance to Robot Personalisation}

Some users may find the level of personalisation required for culturally aware HRI uncomfortable, interpreting the robot's cultural adaptability as unnatural or contrived.

Despite these potential limitations, the framework's design and ongoing development are driven by a commitment to learn from each interaction, thereby gradually improving its competence in engaging with users across diverse cultural identities. 