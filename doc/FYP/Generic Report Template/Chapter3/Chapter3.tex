% !TEX root =  ../Dissertation.tex

\chapter{Objective}

Building upon the observed methodological disparities and theoretical advancements in the field of Human-Robot Interaction (HRI), this study is fuelled by the overarching aim to develop a comprehensive framework that can systematically capture and address the cultural dynamics at play in HRI contexts. The specific objectives of the research are:
\begin{itemize}
\item To Assess Culturally Adaptive behaviours: Examine and assess how culturally adaptive greeting behaviours by robots can enhance their interaction with humans in a multicultural office setting.
\item To Utilize the ROS4HRI Framework: Enhance the emerging ROS4HRI framework to standardize and streamline the experimentation process in HRI research.
\item To Conduct Empirical Studies: Blend qualitative and quantitative research methods, including ROS-based experiments, surveys, and comprehensive user studies, to appraise user perceptions and acceptance of culturally adaptable robotic behaviours.
\item To Inform Robot Design Processes: Produce insights that inform design processes and user experience considerations for robots operating within culturally diverse human environments.
\item To Advance Cultural Robotics: Shed light on how nuanced cultural factors impact the design, function, and acceptance of socially intelligent robots.
\end{itemize}
With these objectives, the study aims to provide actionable insights and empirical benchmarks that can guide the development and deployment of culturally agile robotic systems.

\section{Significance}

This research proposition carries broader implications that transcend individual robotic interactions, with several dimensions of significance:
\begin{itemize}
\item Cultural Sensitivity in Robotic Design: Increase awareness and understanding of the critical role that cultural nuances play in shaping HRI. Inspire the inclusion of cultural considerations in the early stages of robotic design to enhance acceptability and practical utility of robots in diverse environments.
\item Unification of Cultural HRI Research: Establish a unified research methodology that offers consistent interfaces and response mechanisms for culturally informed HRI studies. Foster a collaborative research ecosystem that enables code reusability, facilitates experiment replication, and encourages knowledge sharing across various HRI projects.
\item Empirical Research Contribution: Bridge the current gap between the wealth of theoretical discourse and the lack of empirical evidence concerning cultural influences in HRI. Provide a substantial empirical dataset that supports theoretical claims and encourages continued scholarly investigation.
\item Social and Professional Integration: Enhance the quality of human-robot interactions in professional settings, thereby improving the social integration of robotic agents in the modern workforce. Contribute to the development of culturally inclusive technologies that can accommodate the cultural diversity within global office environments.
\end{itemize}
In doing so, this study aspires to catalyse a paradigm shift in HRI research, where culture is no longer a peripheral consideration, but a pivotal element in crafting more empathetic, responsive, and effective robotic systems for the future.

\section{Contribution}

This research aims to develop a general framework that enables researchers to explore the intersection of robotics and culture. This framework acts as a flexible base on which various parameters relevant to cultural considerations can be adjusted to accommodate a spectrum of studies. Essential to this development is the integration of simulated environments alongside empirical research to glean insights into the cultural dynamics of HRI.

\subsection{Simulation Program Customisation}

A key element of this work involves setting up a dynamic simulation program designed for user-guided research. This program supports parameter alteration, giving researchers the flexibility to design studies around specific cultural subtleties. This customization broadens the scope of HRI research, initiating paths for various applications across different cultural scenarios.

\subsection{Extension of ROS4HRI to Encompass Cultural Aspects}

A significant part of this undertaking involves the enhancement of ROS4HRI, emphasising cultural characteristics. This project aims to insert global cultural aspects like nationality and national culture into the technical sphere of HRI. This advanced feature merges the technical capacity of ROS4HRI with the complex and often subjective facets that shape cultural interaction and perception.
By consolidating these tools under the domain of culture-centric research, this work attempts to encourage exploration and understanding of how robots can be developed more culturally in line and hence, more proficient at working with humans across diverse settings.