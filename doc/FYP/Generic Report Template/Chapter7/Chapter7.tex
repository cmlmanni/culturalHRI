% !TEX root =  ../Dissertation.tex

\chapter{Discussion}

This discussion analyses the results obtained from the implementation and experimentation phases of the culturally aware Human-Robot Interaction (HRI) framework. The discussion aims to situate the findings within the broader field of HRI, considering the complexities of integrating cultural nuances into robotic systems.
Interpretation of Results

The framework effectively demonstrated how robots could adapt behaviors to reflect cultural expressions through differences in gesture, drink choice, language fluency, and speed of movement. These culturally-informed interactions, carefully delineated in the experiments, underscore the framework's potential for enriching HRI research within divergent cultural settings.
While the embodiment of cultural elements through the robot's actions was evident during demonstrations, the subtlety of these behaviors highlights the communication challenges associated with cultural representation in automation. The results indicate a successful proof of concept for culturally-sensitive robot behavior within predefined scenarios. However, they also emphasize the need for clear explication of cultural markers to facilitate better recognition and understanding among a diverse audience.
Contextualization within HRI Research

The integration of culture-specific behaviors in robotics aligns with the evolution of HRI, where personalization and adaptability play an increasingly pivotal role. The study contributes to an emerging corpus of research that investigates not only the functional capabilities of robots but also their social intelligence and empathy within user interactions.
Theoretical Contributions

The research underscores the profound impact of cultural dimensions on human-robot interactions, reaffirming theoretical perspectives on robots as social actors. This work extends the theoretical framework by providing evidence of the feasibility and user reception to culturally aware robots within a controlled experimental environment.
