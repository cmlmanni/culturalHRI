% !TEX root =  ../Dissertation.tex

\chapter{Discussion}

The experiment engaged 15 participants from China (CN), Hong Kong (HK), and the United Kingdom (GB). Notably, one participant, despite being UK-born and residing in the UK, identified as Malaysian (MY), highlighting the complex relationship between geographical origin and cultural identification.

From the experiment, it is apparent that participants' robot preferences varied: some favoured the first robot for its adaptive behaviors, while a majority preferred the second default English robot for its familiar language. These findings underscored the influence of cultural background and language familiarity on human-robot interaction.

Regarding robot behaviours, all participants appreciated the "Conversation Language," with some also favoring the "Proximity" behaviour of the first robot. These insights provide valuable feedback for future cultural robotics design.

Participants' evaluations of robot performance varied, with most rating it as "Good." However, a minority found it "Average," suggesting areas for improvement. The rating of "Poor" due to language barriers highlighted the importance of language accessibility.

The case of the participant identifying as Malaysian (MY), despite being born in and having a significant history in the United Kingdom, raises pertinent considerations regarding the influence of national culture on robot preference. This observation challenges the assumption that national culture alone serves as an adequate criterion for predicting participants' preferences and interaction dynamics in cultural robotics.

The drawback in relying solely on national culture as a determinant of robot preference becomes apparent when considering the complexities of individual identity and cultural hybridity. Participants' cultural backgrounds are often multifaceted and may encompass elements beyond their national origins, such as language, upbringing, education, and personal experiences. In the case of the Malaysian participant, their cultural identification diverged from their national origin, highlighting the inadequacy of using nationality as a sole criterion for understanding preferences in human-robot interaction.

Extending this discussion, it is essential to recognize that cultural identity is fluid and dynamic, shaped by a myriad of factors that extend beyond geographical boundaries. Factors such as globalization, migration, and exposure to diverse cultural influences contribute to the formation of hybrid cultural identities, complicating traditional notions of cultural categorization based on nationality. As such, the assumption that participants' national culture dictates their interaction preferences overlooks the nuanced interplay of various cultural factors that inform individuals' perceptions and behaviors.

Furthermore, the emphasis on national culture as a criterion for robot preference risks perpetuating stereotypes and essentializing cultural identities. Such an approach may overlook the diversity within national cultures and the individual differences that exist among participants from the same cultural background. By oversimplifying cultural identities, researchers run the risk of overlooking important nuances and failing to account for the diversity of preferences and experiences within and across cultural groups.

While the sample size of 15 participants may not be sufficient for generalization, the study framework provides a basis for further research in cultural human-robot interaction (HRI). The Malaysian case underscores the need for additional research to better understand cultural dynamics in HRI. Further investigations with larger and more diverse samples are necessary to validate findings and develop robust frameworks for understanding cultural dynamics in HRI.

Moving forward, researchers in cultural robotics must adopt a more nuanced and inclusive approach to understanding participants' preferences and interaction dynamics. Rather than relying solely on national culture, researchers should consider a broader range of factors, including language proficiency, acculturation level, personal experiences, and individual preferences. By adopting a culturally sensitive and contextually grounded approach, researchers can better capture the complexities of human-robot interaction dynamics and design systems that are inclusive, adaptable, and responsive to diverse cultural contexts and individual preferences.
