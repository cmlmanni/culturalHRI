% !TEX root =  ../Dissertation.tex

\chapter{Future Extension}

The framework, as demonstrated, serves as a foundation for further research into cultural aspects of HRI. Its modular design and adaptable nature make it a valuable platform for future studies exploring intercultural communication and its impact on user-robot dynamics.

\section{Potential Extensions and Applications}

The framework has the potential to accommodate interactions beyond national and geographical cultural boundaries, including variations in gender and race cultures.

\subsection{Gender Culture}

The framework could extend its capabilities to incorporate gender culture. Unlike national cultures, where variations are typically geographically bound, gender cultures exhibit differences within the same geographical area and have unique communication styles, decision-making behaviours, and social norms.

For example, the framework could tailor interactions by adjusting the robot's language and gestures based on the user's gender. It could consider variations in language directness, formality, emotional expressions rooted in different gender cultures.

\subsubsection{Enhancing the Framework's Functionality}

To enhance the framework's functionality, incorporating a gender identity spectrum as part of encoding cultural identity would be beneficial. This integration would facilitate more individualised interactions that consider the user's specific gender identity. For example, for users who identify as non-binary and might not align with traditional gender-based communication styles, robots could adopt a neutral language style and refrain from using gender-coded gestures.

Crucially, the framework must avoid stereotyping or making assumptions based on gender. Instead of treating gender identities as rigid programming rules, knowledge about gender identities should form the basis for respectful interactions. This aspect also needs to be dynamically adjusted and updated based on user interactions, fostering a process of continuous learning.

This mechanism would enable not only the identification and understanding of gender identities but also the nuanced modulations needed for respective interactions. By highlighting continuous learning and adaptability, the framework could cater to a wide range of gender identities and allow for comprehensive, personalised, and respectful interactions in various contexts.

\subsection{Race Culture}

Racial and ethnic cultures often have distinct sets of norms, languages, non-verbal expressions, religious practices, and traditional customs. A truly inclusive HRI framework requires sensitivity to these variations.

The proposed framework can accommodate different racial cultures, similar to nationality. Racial and ethnic cultures have their unique social norms, languages, and non-verbal expressions. The framework's functionality allows robots to adjust their language selection, gestural communication, and hospitality gestures to correspond with specific racial cultures, enabling real-time adaptations to the user's racial-cultural background and enhancing personalisation along with user experiences.

To illustrate, consider the variation in communication styles in different racial or ethnic cultures. In certain Asian cultures, direct eye contact might be viewed as confrontational or disrespectful, particularly when interfacing with individuals of higher status or older age. However, in numerous Western cultures, avoiding eye contact could be seen as an indicator of less confidence or lack of trustworthiness.

Body language and gestures also manifest as highly influenced by race culture. For example, a gesture or body language that may be considered normal or respectful in one culture could be deemed offensive in another.

\subsubsection{Enhancing the Framework's Functionality}

The proposed HRI framework could learn the user's racial or ethnic background and adjust its communication style, eye contact practices, body language, and gestures accordingly. For instance, with a user from a culture that values high-context communication, the robot could become more reliant on non-verbal cues and less dependent on explicit verbal instruction.

Just as with gender culture, the framework must be cautious to avoid stereotyping or making presumptive judgments based on the user's race or ethnicity. It's noteworthy that not all individuals within a certain cultural group will share the same traits or cultural norms.

The recognition and acknowledgment of mixed-race identities is also significant, considering that these individuals may have unique cultural practices stemming from various racial cultures. Similar to gender culture, race culture knowledge should be used as a basis for respectful interaction rather than rigid rules. Once more, the emphasis is on the framework to continue learning and adjusting based on user interactions' feedback.